\documentclass[titlepage, a4paper, 12pt]{article}
\usepackage[swedish]{babel}
\usepackage[utf8]{inputenc}
\usepackage{verbatim}
\usepackage{fancyhdr}
\usepackage{graphicx}
\usepackage{parskip}

% SourceCode
\usepackage{listings}
\usepackage{color}

% Include pdf with multiple pages ex \includepdf[pages=-, nup=2x2]{filename.pdf}
\usepackage[final]{pdfpages}
% Place figures where they should be
\usepackage{float}

% SourceCode
\definecolor{keywordcolor}{rgb}{0.5,0,0.75}
\lstset{
  inputencoding=utf8,
  language=Java,
  extendedchars=true,
  basicstyle=\scriptsize\ttfamily,
  stringstyle=\color{blue},
  commentstyle=\color{red},
  numbers=left,
  firstnumber=auto,
  numberblanklines=true,
  stepnumber=1,
  showstringspaces=false,
  keywordstyle=\color{keywordcolor}
  % identifierstyle=\color{identifiercolor}
}

% Float for text
\floatstyle{ruled}
\newfloat{kod}{H}{lop}
\floatname{kod}{Kodsnutt}

% vars
\def\title{Genetiska algoritmer}
\def\preTitle{Laboration 3}
\def\kurs{Emergenta system, VT-09}

\def\namn{Andreas Jakobsson}
\def\mail{dit06ajs@cs.umu.se}
\def\namnTva{Anton Johansson}
\def\mailTva{dit06ajn@cs.umu.se}

\def\pathtocode{$\sim$dit06ajn/edu/emergenta-system/lab2/src}

\def\handledareEtt{Jonny Pettersson, jonny@cs.umu.se}
\def\handledareTva{Anders Broberg, bopspe@cs.umu.se}

\def\inst{datavetenskap}
\def\dokumentTyp{Laborationsrapport}

\begin{document}
\begin{titlepage}
  \thispagestyle{empty}
  \begin{small}
    \begin{tabular}{@{}p{\textwidth}@{}}
      UMEÅ UNIVERSITET \hfill \today \\
      Institutionen för \inst \\
      \dokumentTyp \\
    \end{tabular}
  \end{small}
  \vspace{10mm}
  \begin{center}
    \LARGE{\preTitle} \\
    \huge{\textbf{\kurs}} \\
    \vspace{10mm}
    \LARGE{\title} \\
    \vspace{15mm}
    \begin{large}
      \namn, \mail \\
      \namnTva, \mailTva\\
      \texttt{\pathtocode}
    \end{large}
    \vfill
    \large{\textbf{Handledare}}\\
    \mbox{\large{\handledareEtt}}
    \mbox{\large{\handledareTva}}
  \end{center}
\end{titlepage}

\newpage
\mbox{}
\vspace{70mm}
\begin{center}
% Dedication goes here
\end{center}
\thispagestyle{empty}
\newpage

\pagestyle{fancy}
\rhead{\today}
\lhead{\footnotesize{\namn, \mail\\\namnTva, \mailTva}}
\chead{}
\lfoot{}
\cfoot{}
\rfoot{}

\cleardoublepage
\newpage
\tableofcontents
\cleardoublepage

% \fancyfoot[LE,RO]{\thepage}
\cfoot{\thepage}
\pagenumbering{arabic}

\section{Problemspecifikation}\label{sec:problemspecifikation}
Laborationen gick ut på att göra ändringar i en befintlig
NetLogo\footnote{http://ccl.northwestern.edu/netlogo/} modell,
''Simple Genetic Algorithm'',som implementerar en enkel genetisk
algoritm. I modellen avgörs vilka individer som får fotplanta sig till
nästa generation med hjälp av en metod som kallas \textit{Tournament
  Selction}. Metoden fungerar genom att slumpmässigt dra tre individer
ur en population, individen med högst \textit{fitness}–värde väljes ut
för fortplantning. En individs \textit{fitness}–värde ska representera
hur pass välanpassad individen är för att lösa ett specifikt problem.

I given modell består problem som ska lösas av att individerna ska
söka efter en sträng av enbart ettor, exempel \verb!"11111"!,
\textit{fitness}–värdet är då antalet ettor i en individs sträng.

Uppgifter som ska lösas (från originalspecifikation):
\begin{itemize}
\item Implementera ytterligare två olika valfira varianter av
  selektion (implementation enligt An Introduction to Genetic
  Algorithms, Melanie Mitchell).
\item Jämför de tre varianterna av selektion med avseende på hur bra
  de bidrar till att så fort som möjligt hitta lösningen
\item Presentera och argumentera för det ni kommer fram till.
\item Reflektera kring laborationen och genetiska algoritmer.
\end{itemize}

% Att göra
% Två valfira varianter av selektion:
% - Tournament selection är implementerad
% - Fitness-Proportionate Selection / Roulette Wheel Selection and Stochastic
%   Universal Sampling
%      reproduce nr of times = individual fitness / mean(population
%      fitness). Dvs de som har bra fittness får fortplanta sig mer än
%      de andra.


\subsection{Frågor som ska behandlas}
I problemspecifikationen finns följande frågor som denna rapport ska
behandla.

\begin{itemize}
\item Vilken selektionsmetod passar bäst för problemet i modellen och varför?
\item För vilken typ av problem passar respektive selektionsmetod?
\item Vilka applikationsområden kan du se för evolutionära algoritmer?
\item För vilken typ av problem ser ni att evolutionära algoritmer
  kommer till mest nytta?
\item Försöka också att sätta in laborationen i ett större sammanhang. 
\end{itemize}
Laborationsspecifikation finns i original på sidan:\\
\verb!http://www.cs.umu.se/kurser/5DV017/VT09/lab/lab3.html!

\section{Användarhandledning}
Källkoden till implementationen som diskuteras i denna rapport finns
att hitta på:

\verb!~dit06ajn/edu/emergenta-system/lab3/src!

Öppna filen i NetLogo för att köra den.

\subsection{Förklaring av användargränssnittet}
Nedan följer en förklaring av de knappar och reglage som förekommer i
användargränssnittet:

\begin{itemize}
\item \textbf{Variabel} - förklaring.
\end{itemize}

\section{Algoritmbeskrivning}

\section{Strategi för testning}

\section{Reflektioner}\label{sec:reflektioner}

Nedan avsnitt beskriver reflektioner som gjorts med avseende på
frågorna från problemspecifikationen.

% Vilken selektionsmetod passar bäst för problemet i modellen och
% varför?
    
% För vilken typ av problem passar respektive selektionsmetod?
    
% Vilka applikationsområden kan du se för evolutionära algoritmer?
    
% För vilken typ av problem ser ni att evolutionära algoritmer kommer
% till mest nytta?

\newpage
\appendix
\pagenumbering{roman}
\section{Källkod}\label{sec:kallkod}
Härefter följer utskrifter från källkoden och andra filer som hör till
denna laboration

\subsection{Flocking.nlogo}\label{app:Flocking.nlogo}
\begin{footnotesize}
  \verbatiminput{../src/Flocking.nlogo}
\end{footnotesize}

\end{document}
