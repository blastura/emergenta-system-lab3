\documentclass[titlepage, a4paper, 12pt]{article}
\usepackage[swedish]{babel}
\usepackage[utf8]{inputenc}
\usepackage{verbatim}
\usepackage{fancyhdr}
\usepackage{graphicx}
\usepackage{parskip}
\usepackage{comment}

% SourceCode
\usepackage{listings}
\usepackage{color}

% Include pdf with multiple pages ex \includepdf[pages=-, nup=2x2]{filename.pdf}
\usepackage[final]{pdfpages}
% Place figures where they should be
\usepackage{float}

% SourceCode
\definecolor{keywordcolor}{rgb}{0.5,0,0.75}
\lstset{
  inputencoding=utf8,
  language=Java,
  extendedchars=true,
  basicstyle=\scriptsize\ttfamily,
  stringstyle=\color{blue},
  commentstyle=\color{red},
  numbers=left,
  firstnumber=auto,
  numberblanklines=true,
  stepnumber=1,
  showstringspaces=false,
  keywordstyle=\color{keywordcolor}
  % identifierstyle=\color{identifiercolor}
}

% Float for text
\floatstyle{ruled}
\newfloat{kod}{H}{lop}
\floatname{kod}{Kodsnutt}

% vars
\def\title{Genetiska algoritmer}
\def\preTitle{Laboration 3}
\def\kurs{Emergenta system, VT-09}

\def\namn{Andreas Jakobsson}
\def\mail{dit06ajs@cs.umu.se}
\def\namnTva{Anton Johansson}
\def\mailTva{dit06ajn@cs.umu.se}

\def\pathtocode{$\sim$dit06ajn/edu/emergenta-system/lab2/src}

\def\handledareEtt{Jonny Pettersson, jonny@cs.umu.se}
\def\handledareTva{Anders Broberg, bopspe@cs.umu.se}

\def\inst{datavetenskap}
\def\dokumentTyp{Laborationsrapport}

\begin{document}
\begin{titlepage}
  \thispagestyle{empty}
  \begin{small}
    \begin{tabular}{@{}p{\textwidth}@{}}
      UMEÅ UNIVERSITET \hfill \today \\
      Institutionen för \inst \\
      \dokumentTyp \\
    \end{tabular}
  \end{small}
  \vspace{10mm}
  \begin{center}
    \LARGE{\preTitle} \\
    \huge{\textbf{\kurs}} \\
    \vspace{10mm}
    \LARGE{\title} \\
    \vspace{15mm}
    \begin{large}
      \namn, \mail \\
      \namnTva, \mailTva\\
      \texttt{\pathtocode}
    \end{large}
    \vfill
    \large{\textbf{Handledare}}\\
    \mbox{\large{\handledareEtt}}
    \mbox{\large{\handledareTva}}
  \end{center}
\end{titlepage}

\newpage
\mbox{}
\vspace{70mm}
\begin{center}
% Dedication goes here
\end{center}
\thispagestyle{empty}
\newpage

\pagestyle{fancy}
\rhead{\today}
\lhead{\footnotesize{\namn, \mail\\\namnTva, \mailTva}}
\chead{}
\lfoot{}
\cfoot{}
\rfoot{}

\cleardoublepage
\newpage
\tableofcontents
\cleardoublepage

% \fancyfoot[LE,RO]{\thepage}
\cfoot{\thepage}
\pagenumbering{arabic}

\section{Problemspecifikation}\label{sec:problemspecifikation}
Laborationen gick ut på att göra ändringar i en befintlig
NetLogo\footnote{http://ccl.northwestern.edu/netlogo/} modell,
''Simple Genetic Algorithm'',som implementerar en enkel genetisk
algoritm. I modellen avgörs vilka individer som får fotplanta sig till
nästa generation med hjälp av en metod som kallas \textit{Tournament
  Selction}. Metoden fungerar genom att slumpmässigt dra tre individer
ur en population, individen med högst \textit{fitness}–värde väljes ut
för fortplantning. En individs \textit{fitness}–värde ska representera
hur pass välanpassad individen är för att lösa ett specifikt problem.

I given modell består problem som ska lösas av att individerna ska
söka efter en sträng av enbart ettor, exempel \verb!"11111"!,
\textit{fitness}–värdet är då antalet ettor i en individs sträng.

Uppgifter som ska lösas (från originalspecifikation):
\begin{itemize}
\item Implementera ytterligare två olika valfira varianter av
  selektion (implementation enligt An Introduction to Genetic
  Algorithms, Melanie Mitchell).
\item Jämför de tre varianterna av selektion med avseende på hur bra
  de bidrar till att så fort som möjligt hitta lösningen
\item Presentera och argumentera för det ni kommer fram till.
\item Reflektera kring laborationen och genetiska algoritmer.
\end{itemize}

% Att göra
% Två valfira varianter av selektion:
% - Tournament selection är implementerad
% - Fitness-Proportionate Selection / Roulette Wheel Selection and Stochastic
%   Universal Sampling
%      reproduce nr of times = individual fitness / mean(population
%      fitness). Dvs de som har bra fittness får fortplanta sig mer än
%      de andra.

\subsection{Frågor som ska behandlas}
I problemspecifikationen finns följande frågor som denna rapport ska
behandla.

\begin{itemize}
\item Vilken selektionsmetod passar bäst för problemet i modellen och varför?
\item För vilken typ av problem passar respektive selektionsmetod?
\item Vilka applikationsområden kan du se för evolutionära algoritmer?
\item För vilken typ av problem ser ni att evolutionära algoritmer
  kommer till mest nytta?
\item Försöka också att sätta in laborationen i ett större sammanhang. 
\end{itemize}
Laborationsspecifikation finns i original på sidan:\\
\verb!http://www.cs.umu.se/kurser/5DV017/VT09/lab/lab3.html!

\section{Användarhandledning}
Källkoden till implementationen lab3.nlogo som diskuteras i denna
rapport finns att hitta på:

\verb!~dit06ajn/edu/emergenta-system/lab3/src!

Öppna filen i NetLogo för att köra den.

\subsection{Förklaring av användargränssnittet}
Nedan följer en förklaring av de knappar och reglage som förekommer i
användargränssnittet:

\begin{itemize}
\item \textbf{population-size} - antalet individer i uppsättningen.
\item \textbf{crossover-rate} - antal procent av populationen som ska användas för sexuell reproduktion.
\item \textbf{mutation-rate} - antal procent av resterande(???) individer som kan muteras.
\item \textbf{go} - denna knapp sätter igång förloppet. Metoden upprepas tills dess att målet är uppfyllt (enbart ettor, enbart vita sträck). Setup måste köras en gång innan denna knapp får användas.
\item \textbf{setup} - initierar parametrar till simuleringen.
\item \textbf{step} - anropar proceduren go endast en gång.
\item \textbf{} - XXXXnågotmer?

\begin{itemize}
\item \textbf{Variabel} - förklaring.
\end{itemize}

\section{Algoritmbeskrivning}
Nedan följer förklaring över de selektionsmetoder som är
implementerade i denna laboration. Alla selektionsmetoder
implementerades utifrån beskrivningar och givna i boken \textit{An
  introduction to genetic algorithms}, \cite{gen-intro}.

\subsection{Roulette Wheel}\label{sec:roulette-wheel}
Selektionsalgoritmen \textit{Roulette Wheel} har fått sitt namn på
grund av att metoden kan liknas vid att snurra ett Roulette–hjul där
vissa individer har större sannolikhet att väljas baserat på dess
\textit{fitness}.

För algoritmen räknas ett värde, \textit{expected-value}, som ger en
fingervisning om hur välanpassad en individ är gentemot de andra
individerna i populationen. Detta värde räknas ut för varje individ
genom att dela dess \textit{fitness}–värde med medelvärdet av hela
populationens \textit{fitness}–värden.

Följande metod följdes för att välja individer för sexuell
reproduktion:

\begin{enumerate}
\item Summera totala antalet individer i populationen, kalla denna
  summa \textit{T}.
\item Upprepa följande steg antalet gånger som ett urval ska göras. I
  denna laboration fall bestäms detta av parametern
  \textit{crossover-rate}.
  \begin{itemize}
  \item Välj ett slumpmässig tal \textit{r} mellan 0 och
    \textit{T}. Iterera över individerna i populationen, summera
    värdet på \textit{expected-value} tills summan är större än eller
    lika med \textit{r}. Välj ut individen vars värde går över denna gräns.
  \end{itemize}
\end{enumerate}

\subsection{Roulette Wheel with Sigma Scaling}
Selektionsmetoden \textit{Roulette Wheel with Sigma Scaling} används
exakt som \textit{Roulette Wheel}, se avsnitt
\ref{sec:roulette-wheel}, förutom att individernas
\textit{expected-value} räknas ut enligt följande formel, (direkt
tagen ur \cite{gen-intro}):

\begin{displaymath}
  \textrm{ExpVal(i,t)} = \left\{ \begin{array}{ll}
      1 + \frac{f(i) - \overline{f}(t)}{2\sigma(t)} & \textrm{om } \sigma(t) \neq 0 \\
      1.0 & \textrm{om } \sigma(t) = 0
    \end{array} \right.
\end{displaymath}
  
Här avser $f(i)$ \textit{fittness} för individ $i$, $\overline{f}(t)$
avser medelvärdet för populationens \textit{fitness}–värde vid tiden
$t$ och $\sigma(t)$ är standardavvikelsen vid tiden $t$.

% TODO: forsätt diskussion ey.

  \subsection{Steady State}
% TODO piece of cake
  
\section{Strategi för testning}

\section{Metod för testning}
De implementerade selektionsmetoderna utvärderas genom att jämföra antalet omgångar, hädan efter kallade ticks, det tar för metoden att hitta en fullständig lösning. För att generera testdata gjordes ett test-script som kan aktiveras från användargränssnittet. Test-scriptet sätter programmets parametrar och itererar varje selektionsmetod ett förinställt antal gånger. Antalet ticks som krävs för att hitta en lösning sparas i en textfil. Efter att en sorts selektionsmetod har körts klart avslutas filen med medelvärdet och standardavvikelsen för den insamlade datan. Därefter startar en ny fil där data från nästa selektionsmetod sparas. När alla valda selektionsmetoder har kört klart avslutas skriptet. Sammanställning av resultaten kan ses under avsnitt (xxxx). 
\section{Resultat}
% Tabell och sånt

\section{Reflektioner}\label{sec:reflektioner}

Nedan avsnitt beskriver reflektioner som gjorts med avseende på
frågorna från problemspecifikationen.

% Anton: Vilken selektionsmetod passar bäst för problemet i modellen och
% varför?
Eftersom resultatet inte pekar på större skillnader i tidsåtgång är det svårt att resonera om skillnader mellan prestandan av selektionsmetoderna som löste uppgiften. Den redan implementerade Tournament-metoden står ändå ut med tanke på den enkla implementationen och kan därför ses som en segrare. Ett förslag till förbättring till denna metod vore att lägga till så kallad Elitism-procedur (XXX källa paper?) för att låta en del av de bästa individerna gå vidare direkt till nästa generation och alltså eliminera risken av att de blir undansparkade av slumpen.
    
% För vilken typ av problem passar respektive selektionsmetod?
Med avseende på tidsåtgång förlorar Roulette Wheel som sällan hittade rätt lösning men var snabb inledningsvis (XXXX sant?) på att höja den genomsnittliga fittnessen (XXXX fittnessen??). Problemet var att utvecklingen av populationen avstannade (premature konvergence) eftersom variansen blev för liten och de med bättre fitness tappade övertaget. Denna selektionsmetod lämpar sig alltså bara inledningsvis i evolutionen eller om ett perfekt resultat inte är nödvändigt.


    
% Anton: Vilka applikationsområden kan du se för evolutionära algoritmer?
    
% Andreas: För vilken typ av problem ser ni att evolutionära algoritmer kommer
% till mest nytta?
I laboration 1 där termiter skulle flytta träbitar till en enda hög fanns en del parametrar att skruva på. Det fanns även flera önskvärda beteenden som skulle uppnås. Förutom att alla träbitar skulle ligga samlat ska de även forma en cirkulär samling. Dessutom konvergerar högen snabbare om två konkurrerande myrsorter bygger sina högar åtskiljt. Om en uppsättning parametrar uppfyller dessa krav kan lösningen anses fullständig, alltså en sträng med ettor. Denna parameterskruvning skulle kunna göras med hjälp av evolutionära algoritmer.

Evolutionära algoritmer borde passa bra till just parameterskruvning eftersom rätt kombination av parametrarna ej behöver vara trivial och behöver testas fram och utvärderas.

\bibliographystyle{alpha}
\bibliography{books}

\newpage
\appendix
\pagenumbering{roman}
\section{Källkod}\label{sec:kallkod}
Härefter följer utskrifter från källkoden och andra filer som hör till
denna laboration

\subsection{Flocking.nlogo}\label{app:Flocking.nlogo}
\begin{footnotesize}
  \verbatiminput{../src/Flocking.nlogo}
\end{footnotesize}

\end{document}
